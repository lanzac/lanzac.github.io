%-------------------------
% Resume in LateX
% Author : Sourabh Bajaj
% License : MIT
%------------------------

\documentclass[a4paper,11pt]{article}

\usepackage{latexsym}
\usepackage[empty]{fullpage}
\usepackage{titlesec}
\usepackage{marvosym}
\usepackage[usenames,dvipsnames]{color}
\usepackage{verbatim}
\usepackage{enumitem}
\usepackage[hidelinks]{hyperref}
\usepackage{fancyhdr}
\usepackage[english]{babel}
\usepackage{tabularx}
\usepackage{multicol}
\usepackage{pifont}
\usepackage{ifthen}
\usepackage{fontspec}
\usepackage{luatexja-fontspec}

% p.62 of: https://mirrors.ibiblio.org/CTAN/macros/latex/contrib/biblatex/doc/biblatex.pdf
\usepackage[
  style=numeric-comp,
  doi=true,
  isbn=false,
  url=false,
  sorting=none,
  backend=biber
]{biblatex}

% Hide fileds "pages", "volume", "number"
\AtEveryBibitem{
  \clearfield{pages} % Masque le champ "pages"
  \clearfield{volume} % Masque le champ "volume"
  \clearfield{number} % Masque le champ "number"
}


\DeclareFieldFormat{doi}{
  \mbox{\href{https://doi.org/#1}{\texttt{DOI:#1}}}
}


% https://tex.stackexchange.com/questions/401020/how-to-remove-the-language-information-in-a-bibliography
\AtEveryBibitem{\clearlist{language}}
% https://tex.stackexchange.com/questions/10682/suppress-in-biblatex
\renewbibmacro{in:}{}
\addbibresource{assets/tex/publications.bib}
\usepackage{csquotes}

\pagestyle{fancy}
\fancyhf{} % Clear all header and footer fields
\fancyfoot{}
\renewcommand{\headrulewidth}{0pt}
\renewcommand{\footrulewidth}{0pt}

% Adjust margins
\addtolength{\oddsidemargin}{-0.5in}
\addtolength{\evensidemargin}{-0.5in}
\addtolength{\textwidth}{1in}
\addtolength{\topmargin}{-.5in}
\addtolength{\textheight}{1.0in}

\urlstyle{same}

\raggedbottom
\raggedright
\setlength{\tabcolsep}{0in}

% fc-list  | cut -d\  -f2-99 | cut -d: -f1 | sort -u
% Linux Libertine O
\setmainfont{Open Sans}[
    BoldFont = {Open Sans SemiBold},
    SmallCapsFont = {TeX Gyre Termes},
    SmallCapsFeatures = {Letters=SmallCaps}
]

% Sections formatting
\titleformat{\section}{
  \vspace{-5pt}\raggedright\bfseries\large
}{}{0em}{}[\color{black}\titlerule \vspace{-5pt}]

%-------------------------
% Custom commands
% \newcommand{\resumeItem}[2]{
%   \item\small{
%     \textbf{#1}{: #2 \vspace{-2pt}}
%   }
% }
\newcommand{\resumeItem}[1]{
  \item \begin{minipage}[t]{0.98\textwidth}
    \raggedright #1
  \end{minipage}\vspace{-2pt}
}

% Just in case someone needs a heading that does not need to be in a list
\newcommand{\resumeHeading}[4]{
    \begin{tabular*}{0.99\textwidth}[t]{l@{\extracolsep{\fill}}r}
      \textbf{#1} & #2 \\
      \textit{#3} & \textit{#4} \\
    \end{tabular*}\vspace{-5pt}
}

\newcommand{\resumeSubheadingBis}[4]{
  \vspace{-1pt}\item
    \begin{tabular*}{\textwidth}[t]{l@{\extracolsep{\fill}}r}
      \textbf{#1} & \textit{#2} \\
      \textit{#3}
    \end{tabular*}
    \begin{minipage}[t]{0.85\textwidth} #4 \end{minipage} \vspace{-2pt}
    \vspace{2pt}
}


\newcommand{\resumeSubSubheading}[2]{
    \begin{tabular*}{0.97\textwidth}{l@{\extracolsep{\fill}}r}
      \textit{#1} & \textit{#2} \\
    \end{tabular*}\vspace{-5pt}
}

\newcommand{\resumeSubItem}[2]{\resumeItem{#1}{#2}\vspace{-4pt}}

\renewcommand{\labelitemii}{$\circ$}

\newcommand{\resumeSubHeadingListStart}{\begin{itemize}[leftmargin=0pt, label={}, itemsep=0pt]}
\newcommand{\resumeSubHeadingListEnd}{\end{itemize}}
\newcommand{\resumeItemListStart}{\begin{itemize}[leftmargin=*, label={\sbt}]}
\newcommand{\resumeItemListEnd}{\end{itemize}\vspace{-5pt}}



% https://www.johndcook.com/blog/2011/10/18/typesetting-c-in-latex/
\usepackage{relsize}
\newcommand{\Cs}{C\nolinebreak\hspace{-.15em}\textscale{0.9}{\symbol{"266F}}}
\newcommand{\Cpp}{C\texttt{++}}

\newcommand{\sbt}{\,\begin{picture}(-1,1)(-1,-3)\circle*{3}\end{picture}\ }

\newcommand{\formatDate}[2]{
  \ifthenelse{\equal{#1}{#2}}
    {#1}
    {#1 -- #2}
}

\setlength{\footskip}{4.1pt}

\usepackage[none]{hyphenat} % Désactive les césures globalement
\raggedright               % Texte aligné à gauche, sans justification

\setlength{\spaceskip}{3pt}
\setlength{\xspaceskip}{3pt}



%-------------------------------------------
%%%%%%  CV STARTS HERE  %%%%%%%%%%%%%%%%%%%%%%%%%%%%


\begin{document}

%----------HEADING-----------------
\begin{tabular*}{\textwidth}{l@{\extracolsep{\fill}}r}
  \textbf{\href{https://lanzac.github.io}{\Large André Lanrezac}} & \href{mailto:a.lanrezac@gmail.com}{a.lanrezac@gmail.com}\\
  Bioinformaticien | Développeur scientifique & \href{https://lanzac.github.io}{lanzac.github.io} \\
  % Mobile: \href{tel:}{}
\end{tabular*}

%--------SKILLS------------
  \section{Compétences}
  \begin{tabularx}{\textwidth}[t]{@{}lX@{}}
    \textbf{Programmation: } & \sffamily{ Python3, \Cpp, \Cs, Bash, C, HTML, CSS } \\
    \textbf{Technologies: } & \sffamily{ Git, CI/CD (GitHub Actions), Docker, Unity, GDB, LLDB, CMake, MongoDB, Streamlit } \\
    \textbf{Langues: } & \sffamily{ Français (natif), Anglais (maîtrisé), Mandarin (débutant) } \\
    \end{tabularx}

%-----------EXPERIENCE-----------------
\newcommand{\resumeSubheadingExperience}[4]{
  \vspace{-1pt}\item
    \begin{tabular*}{\textwidth}[t]{l@{\extracolsep{\fill}}r}
      \textbf{#1} & #2 \\
      #3, #4 & \\
    \end{tabular*}\vspace{-7pt}
}

\section{Expériences professionnelles}
  \resumeSubHeadingListStart
    \resumeSubheadingExperience
      {Assistant de recherche \& Ingénieur de recherche}{\formatDate{2019}{2024}}
      { \href{http://www-lbt.ibpc.fr/}{Laboratoire de Biochimie Théorique, IBPC · CNRS} }{Paris}
      \resumeItemListStart
        \resumeItem{Auteur principal du premier article détaillé sur les simulations moléculaires interactives (IMS) \cite{lanrezac2022}, synthétisant des décennies de recherches et publié dans une revue à fort impact}
        \resumeItem{Amélioration d'une interface de visualisation moléculaire 3D (Unity, \Cs) pour l'interaction et la visualisation en temps réel des études d'insertion de protéines dans des membranes \cite{lanrezac2022a,lanrezac2023a}}
        \resumeItem{Standardisation des workflows multi-threadés (\Cpp, Python) pour contrôler les simulations en temps réel, permettant un échange fluide de données et une analyse directe des métriques}
        \resumeItem{Développement d'algorithmes Python adaptant les modèles implicites tout-atomes à Martini3 en gros grain, permettant des simulations interactives avec des géométries membranaires courbées \cite{lanrezac2023a}}
        \resumeItem{Création d'un pipeline CI/CD (GitHub Actions, Docker) automatisant les versions du moteur de simulation, résolvant des problèmes de dépendances et améliorant l'accessibilité du logiciel}
      \resumeItemListEnd
      \vspace{5pt}
    \resumeSubheadingExperience
      {Stagiaire en conception de médicaments en bioinformatique}{\formatDate{Fev 2019}{Juin 2019}}
      { \href{https://impmc.sorbonne-universite.fr/fr/equipes/biophysique_et_bioinformatique.html}{Équipe de bioinformatique et de biophysique, IMPMC · Sorbonne université} }{Paris}
      \resumeItemListStart
        \resumeItem{Développement d'une méthode d'amarrage de peptides cycliques en combinant Autodock avec des simulations REMD dans GROMACS, et comparaison des résultats avec l'outil d'amarrage HADDOCK}
        \resumeItem{Conception manuelle de peptides cycliques avec UCSF-Chimera}
      \resumeItemListEnd
      \vspace{5pt}
    \resumeSubheadingExperience
      {Stagiaire en analyse structurale en bioinformatique}{\formatDate{Juillet 2018}{Juillet 2018}}
      { \href{https://isyeb.mnhn.fr/fr/atelier-de-bio-informatique-384}{Bioinformatics Research Group, ISYEB · Sorbonne université} }{Paris}
      \resumeItemListStart
        \resumeItem{Optimisation d'une recherche de similarité protéique en réimplémentant l'algorithme de Bellman (Python) pour identifier le plus long chemin à travers des motifs communs, et en développant des algorithmes heuristiques pour un alignement plus rapide}
      \resumeItemListEnd
      \vspace{5pt}
    \resumeSubHeadingListEnd

% --------Multiple Positions Heading------------
  %  \resumeSubSubheading
  %   {Software Engineer I}{Oct 2014 -- Sep 2016}
  %   \resumeItemListStart
  %      \resumeItem{Apache Beam}
  %        {Apache Beam is a unified model for defining both batch and streaming data-parallel processing pipelines}
  %   \resumeItemListEnd

%-------------------------------------------

%-----------PROJECTS-----------------
% \section{Projets}
% \resumeSubHeadingListStart
%   %   \resumeSubheadingBis
%     {}{ --  }
%     {}
%     {}
%   % \resumeSubHeadingListEnd


%-----------EDUCATION-----------------
\newcommand{\resumeSubheadingEducation}[4]{
  \vspace{-1pt}\item
    \begin{tabular*}{\textwidth}[t]{l@{\extracolsep{\fill}}r}
      \textbf{#1} -- #2 & #3 \\
      \ifx&#4&\else #4 \\ \fi
    \end{tabular*}\vspace{-7pt}
}

\section{Formation}
  \resumeSubHeadingListStart
    \resumeSubheadingEducation
      {Université Paris Cité}{Doctorat en Bioinformatique}{ 2023}
      { Directeur: Dr Marc Baaden}
      
    \resumeSubheadingEducation
        {Sorbonne Université, Paris}{Licence en Biologie \& Master en Bioinformatique}{ 2019}
      {}
      
    \resumeSubHeadingListEnd

%--------PUBLICATIONS------------
\nocite{*}
\printbibliography[title=Publications, nottype=cited]
%-------------------------------------------

\end{document}